% Титульный лист (ГОСТ Р 7.0.11-2001, 5.1)
\thispagestyle{empty}%
\begin{center}%
	\large ФГБОУ ВО
	
	\large \MakeUppercase{\thesisOrganization}
	\begin{figure}[ht] 
		\center
		\includegraphics [height=30pt] {mpei_label}
		\includegraphics [height=30pt, width=\textwidth] {mpei}
		\noindent\rule{\textwidth}{2pt}
	\end{figure}
	
	\large\kafedra
\end{center}%

\vspace{0pt plus4fill} %число перед fill = кратность относительно некоторого расстояния fill, кусками которого заполнены пустые места
%

\vspace{0pt plus7fill} %число перед fill = кратность относительно некоторого расстояния fill, кусками которого заполнены пустые места
\begin{center}%
	\Large\MakeUppercase{Курсовая работа}
	
	\large По курсу <<\courseName>>

	\vspace{0pt plus2fill} %число перед fill = кратность относительно некоторого расстояния fill, кусками которого заполнены пустые места

	%\thesisDegree
\end{center}%
%
\vspace{0pt plus4fill} %число перед fill = кратность относительно некоторого расстояния fill, кусками которого заполнены пустые места
\begin{flushright}%
	\begin{table} [htbp]% Пример записи таблицы с номером, но без отображаемого наименования
		\raggedleft
		\parbox{9cm}{% чтобы лучше смотрелось, подбирается самостоятельно
			\begin{tabular}{  r  l }
				Группа:	& А-13м-16 \\
				Студент: 	& \thesisAuthorShort \\
				Преподаватель:	& \supervisorFioShort \\
			\end{tabular}%
		}
	\end{table}
%\supervisorRegalia
\end{flushright}%
%
\vspace{0pt plus4fill} %число перед fill = кратность относительно некоторого расстояния fill, кусками которого заполнены пустые места
\begin{center}%
{\thesisCity~--- \thesisYear}
\end{center}%
\newpage
