\chapter*{Введение}							% Заголовок
\addcontentsline{toc}{chapter}{Введение}	% Добавляем его в оглавление

\newcommand{\actuality}{}
\newcommand{\progress}{}
\newcommand{\aim}{{\textbf\aimTXT}}
\newcommand{\tasks}{\textbf{\tasksTXT}}
\newcommand{\novelty}{\textbf{\noveltyTXT}}
\newcommand{\influence}{\textbf{\influenceTXT}}
\newcommand{\methods}{\textbf{\methodsTXT}}
\newcommand{\defpositions}{\textbf{\defpositionsTXT}}
\newcommand{\reliability}{\textbf{\reliabilityTXT}}
\newcommand{\probation}{\textbf{\probationTXT}}
\newcommand{\contribution}{\textbf{\contributionTXT}}
\newcommand{\publications}{\textbf{\publicationsTXT}}


{\actuality} Предметно-ориентированный язык (англ. Domain-specific language, DSL) -- язык программирования, специализированный для конкретной области применения. Построение такого языка и/или его структура данных отражают специфику решаемых с его помощью задач.

Использование DSL имеет ряд преимуществ:
\begin{itemize}
	\item на этапе проектирования дает возможность создания решений в терминах предметной области, благодаря чему специалисты в данной предметной области могут создавать и модифицировать DSL программы;
	\item решение проблемы с помощью DSL происходит на соответствующем уровне абстракции. Это позволяет экспертам в предметной области понимать и верифицировать DSL-программы;
	\item программы, написанные с использованием DSL лаконичны. Написание DSL с использованием терминов предметной области дает возможность в дальнейшем читать программу достаточно легко;
	\item происходит повышение надежности, эффективности и качества сопровождения. Поскольку на уровне модели операции осуществлять легче, они более эффективны и подвержены меньшему количеству ошибок, чем те же операции на уровне кода;
	\item DSL позволяет на уровне абстракции, соответствующему домену, проводить оптимизацию и валидацию;
	описание домена на одном уровне абстракции можно потом преобразовать в более низкий уровень с подробной детализацией. Таким образом можно дополнять модель на разных этапах разработки.
\end{itemize}

{\aim} Спроектировать язык для описания динамической вычислительной сети.

Для~достижения поставленной цели необходимо было решить следующие {\tasks}:
\begin{enumerate}
  \item Изучить способы реализации DSL.
  \item Составить концептуальную модель языка
  \item Реализовать язык в системе Meta Programming System
  \item Проверить результат на соответствия ограничениям.
\end{enumerate}
    

 % Характеристика работы по структуре во введении и в автореферате не отличается (ГОСТ Р 7.0.11, пункты 5.3.1 и 9.2.1), потому её загружаем из одного и того же внешнего файла, предварительно задав форму выделения некоторым параметрам