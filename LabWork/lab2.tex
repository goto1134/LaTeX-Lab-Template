\section{Цель работы}
	Изучение основ работы с генетическими алгоритмами.
\section{Порядок выполенения работы}
	\subsection{Задача}
		Задача основана на другой задаче, описанной на сайте \href{http://www.codeabbey.com/index/task_view/random-search-optimization}{www.codeabbey.com}.

		Дана некоторая беспилотная транспортная машина бесконечно малого размера такая, что может находится в одной точке с некоторым контейнером. Контейнер и машина связаны эластической пружиной так, что когда машина удаляется на достаточное расстояние от контейнера, сила натяжения начинает притягивать коробку к машине.
		
		Когда машина достигает 100м отметки, она останавливается, но коробка может перемещаться дальше.
		
		Каждые 20м есть возможность изменить скорость машины не более, чем на 3м/c, из-за опасности возгорания двигателя. Таким образом 100-метровый путь разделён на 5 частей и на каждой из частей машина движется с постоянной скоростью. На границе скорость машины резко изменяется.
		
		Алгоритм симуляции движения известен. Необходимо найти 5 значений скорости, для которых время от момента начала движения до момента остановки контейнера в точке 100м будет минимально.
		
		
	\subsection{Инструменты разработки}
		Для решения задачи выбран язык Java и библиотека \href{http://jenetics.io/}{Jenetics}.  На данный момент Jenetics явяется самой актуальной, т.к. использует все последние возможности языка Java, что сильно упрощает процесс создания алгоритма.L Руководство по использованию библиотеки доступно по \href{http://jenetics.io/manual/manual-3.6.0.pdf}{ссылке}.
		
	\subsection{Генотип}
		Т.к. параметры функции - 5 вещественных чисел с разными множествами допустимых значений, использован генотип, состоящий из 5-и хромосом, содержащих по 1 вещественному гену.
		
		\begin{ListingEnv}[H]% буква H означает Here, ставим здесь,
			% элементы, которые нежелательно разрывать обычно не ставят
			% посреди страницы: вместо H используется t (top, сверху страницы),
			% или b (bottom) или p (page, на отдельной странице)
			%    \captionsetup{format=tablenocaption}% должен стоять до самого caption
			%    \thisfloatsetup{\capposition=top}%
			\caption{Описание генотипа}
			% далее метка для ссылки:
			\label{list:hwbeauty}
			% окружение учитывает пробелы и табляции и приеняет их в сответсвии с настройкми
			\begin{lstlisting}[language={Java}]
			final Factory<Genotype<DoubleGene>> genotypeScalarFactory = Genotype.of(DoubleChromosome.of(0, 3, 1)
				, DoubleChromosome.of(0, 6, 1)
				, DoubleChromosome.of(0, 9, 1)
				, DoubleChromosome.of(0, 6, 1)
				, DoubleChromosome.of(0, 3, 1)
			);
			\end{lstlisting}
		\end{ListingEnv}%
	
	\subsection{}
	