	\section{Цель работы}
		Научиться формулировать постановку задачи на разработку компонента распределенной системы и фиксировать требования.
	\section{Постановка задачи}
	
	Необходимо разработать модуль, реализующий логику переключения светодиодного светофора для автомобилей. При выполнении задания считать, что светофор один, установлен на однополосной дороге и закреплен на столбе так, чтобы быть видным водителям транспортных средств.
	
	
	Логика должна обеспечивать следующий цикл переключений сигналов:	
	\begin{itemize}
		\item 40 секунд зеленый;
		\item 4 секунды желтый;
		\item 30 секунд красный.
	\end{itemize}

	Управляющая логика, которую следует разработать, должна взаимодействовать с исполнительным модулем, умеющим взаимодействовать с контроллером светофора и контролирующей логикой. Модуль-исполнитель умеет устанавливать на светофоре нужный сигнал — зеленый, красный, желтый. Он не умеет организовывать ожиданий. Т.е., нельзя сказать ему: <<показывай зеленый сигнал 40 секунд>>. Время должна контролировать управляющая логика. Контролирующая логика решает задачу включения цикла работы управляющей логики и получает от нее сообщения о проблемах.
	
	При выполнении задания следует считать, что аппаратура светофора обеспечивает:
	\begin{itemize}
		\item мигающий желтый сигнал при включении;
		\item мигающий желтый сигнал в случае аварии контроллера (автоматически включается когда не поступают воздействия от контроллера).
	\end{itemize}

	Модуль-исполнитель пассивен по своей природе и обеспечивает реакцию на команды, никогда не начиная обмены по своей инициативе. Управляющая логика должна в нужные моменты передавать через коммуникационную среду сообщения-команды исполнительному модулю и ожидать от него ответа об исполнении. Для установки сигнала нужного света нужно отправить исполнительному модулю сообщение \textit{SetLight}. Ниже приведен протокол модуля-исполнителя:
	
	\begin{lstlisting}
Enumerations ; Перечислимые типы
ColorType                | Цвет светофора
1 : 3
Red    : 1           |- Красный цвет 
Yellow : 2           |- Желтый цвет
Green  : 3           |- Зеленый цвет
Messages ; Сообщение
SetLight : 1             | Установить новый сигнал
1 : 1
Color : ColorType    |- Требуемый цвет сигнала
Success : 2              | Команда контроллеру отработана
1 : 0
Fail : 3                 | Ошибка в работе контроллера (авария)
1 : 0 
	\end{lstlisting}
	
	После отправки исполнителю команды (сообщения \textit{SetLight}) следует учесть три возможных варианта дальнейшего развития событий:
	\begin{enumerate}
		\item 	исполнитель успешно отрабатывает запрос и светофор показывает нужный сигнал — ситуация, в которой получено ответное сообщение \textit{Success};
		\item исполнитель обнаружил, что не может обработать запрос и сообщает об аварии контроллера — ситуация, в которой получено ответное сообщение \textit{Fail};
		\item управляющая логика отправила запрос, но ничего не получила в ответ — ситуация, которую можно «поймать» с помощью тайм-аута нахождения в состоянии ожидания ответа от модуля-исполнителя.
	\end{enumerate}
	
	Варианты 2 и 3 при выполнении задания считать аварийными. Обрабатывая эту ситуацию, следует остановить работу управляющей логики на паузу, отправив вышестоящему уровню (контролирующей логике) оповещение \textit{ProblemDetected}. В состоянии паузы управляющая логика должна уметь воспринимать команды «выгрузиться» (сообщение \textit{Shutdown}) и «возобновить работу» (сообщение Start). Изначально управляющая логика должна находиться в состоянии паузы.
	\begin{lstlisting}
Enumerations ; Перечислимые типы
Messages ; Сообщения
Start : 1           | Включить цикл работы светофора
1 : 0
Shutdown : 2        | Выгрузить управляющую логику
1 : 0
ProblemDetected : 3 | Проблема в работе светофора
1 : 0
	\end{lstlisting}
	
При реализации следует также использовать системные сообщения среды исполнения, определенные протоколом обмена \textit{Kernel}.

	\begin{lstlisting}
Messages ; Сообщения
TimeOut : 1                                 | Сообщение TimeOut. Рассылается машинами состояний в случае превышения допустимого интервала нахождения в определенном состоянии.
1 : 0
ModuleFailedOnEvent : 2                     | Модуль не смог обработать сообщение
1 : 3
EventKind : int                         | - Тип, события, при реакции на которое произошла ошибка
ExceptionKind : string                  | - Тип ошибки (исключения)
ExceptionMessage : string               | - Сообщение об ошибке
Start : 3                                   | Команда инициализации
1 : 0
Stop : 4                                    | Команда остановки
1 : 0 ; Поколение 1, число аргументов 0
StateChanged : 5                            | Событие, присылаемое в момент изменения состояния конечного автомата
1 : 4
MachineName : string                    | - Имя конечного автомата, перешедшего в новое состояние
OldState : string                       | - Старое состояние машины
NewState : string                       | - Новое состояние машины
Event : int                             | - Событие, активировавшее переход
	\end{lstlisting}
	
	\section{Требования к решению}
	Для успешного прохождения тестирования кандидат должен предоставить комплект материалов:
	
	\begin{itemize}
		\item Сценарии работы модуля управляющей логики
		\item Чек-лист
		\item Модель конечного автомата
		\item Код
	\end{itemize}
	
	\section{Вывод}
		Поставлена задача на разработку компонента распределенной системы и зафиксированы требования.