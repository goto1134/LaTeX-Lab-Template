\section{Цель работы}
	Научиться разрабатывать обработчики переходов в конечном автомате на языке C++
		
\section{Порядок выполенения работы}
	\subsection{Генерация файлов протокола}
		Собраны ресурсы модуля с помощью  Microkernel IDE. Заголовочные файлы скопированы в заготовку проекта в Visual Studio.
		\lstinputlisting[caption={Протокол контроллера}, language=C++]{listings/Uniteller.CrossroadController.h}
		\lstinputlisting[caption={Протокол светофора}, language=C++]{listings/Uniteller.DeviceController.h}
		\lstinputlisting[caption={Простокол основной логики}, language=C++]{listings/Uniteller.MainLogic.h}
	\subsection{Карта обработчиков}
		Заполнена карта обработчиков - добавлено нужное число регистраций функций переходов
		\lstinputlisting[caption={Заголовочный файл машины состояний}, language=C++]{listings/machine.h}
		
	\subsection{Реализация обработчиков сообщений}
		Реализованы тела обработчиков.
		\lstinputlisting[caption={Реализация методов машины состояний}, language=C++]{listings/machine.cpp}
	
	\subsection{Сборка модуля}
		Проект собран и создан файл динамической библиотеки.
		\lstinputlisting[caption={Вывод сборки}]{listings/build.txt}

\section{Вывод}
	Разработаны обработчики переходов в конечном автомате на языке C++.