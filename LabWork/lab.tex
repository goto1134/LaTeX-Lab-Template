	\section{Цель работы}
	Разработать библиотеку для реализации клиента программы SIMPR на языке Java.
		
	\section{Порядок выполенения работы}
 		\subsection{Выбор технологий для реализации}
 			В качестве языка для реализации библиотеки выбрана Java версии 8. 
		   
			Т.к. SIMPR осуществляет взаимодействие с клиентом посредством Win32 API, требуется использование библиотеки, позволяющей осуществить такое взаимодействие. 
			В качестве основы для такого взаимодействия выбрана библиотека \href{https://github.com/jnr/jnr-ffi}{jnr-ffi}.
			Данная библиотека позволяет описать все нужные методы для работы с динамически подгружаемыми библиотеками в специального вида Java-интерфейсах. 
			Таким образом можно построить необходимую прослойку для обмена сообщениями.

	 	\subsection{Реализация}
	 		Результат реализации можно получить на сайте \href{https://github.com/goto1134/simpr-java-client}{GitHub} (https://github.com/goto1134/simpr-java-client). 
	 		Там же можно посмотреть исходный код проекта и предложить исправления.
	 		
	 	\subsection{Инструкция}
	 		\begin{enumerate}
	 			\item Реализовать интерфейс $ com.github.goto1134.simpr.SimprClient $.
	 				\begin{ListingEnv}[!h]% настройки floating аналогичны окружению figure
%	 					\captiondelim{ } % разделитель идентификатора с номером от наименования
	 					\caption{Пример реализации интерфейса SimprClient}
	 					% далее метка для ссылки:
	 					\label{list:client}
	 					% окружение учитывает пробелы и табуляции и применяет их в сответсвии с настройками
	 					\begin{lstlisting}[language={Java}]
public class TestApp
		implements SimprClient {
	@Override
	public boolean getConditionValue(int tableIndex, int conditionIndex) {
		Alert alert = new Alert(CONFIRMATION, "Is condition " + tableIndex + " " + conditionIndex + " true?", ButtonType.YES, ButtonType.NO);
		alert.setTitle("Checking condition");
		Optional<ButtonType> buttonType = alert.showAndWait();
		return buttonType.get() == ButtonType.YES;
	}
	
	@Override
	public void onEndStateReached(int tableIndex) {
		Alert alert = new Alert(INFORMATION, "End state reached in table " + tableIndex);
		alert.show();
	}

	@Override
	public boolean performEvent(int tableIndex, int event) {
		System.out.println("event " + tableIndex + " " + event + " occured");
		return true;
	}
}
	 					\end{lstlisting}
	 				\end{ListingEnv}%
	 			\item При запуске приложения создать экземпляр 
	 			$ com.github.goto1134.simpr.SimprMessageHandler $, передав в него реализованный интерфейс.
	 			\begin{ListingEnv}[!h]% настройки floating аналогичны окружению figure
	 				%	 					\captiondelim{ } % разделитель идентификатора с номером от наименования
	 				\caption{Пример кода, запускающего приложение}
	 				% далее метка для ссылки:
	 				\label{list:handler}
	 				% окружение учитывает пробелы и табуляции и применяет их в сответсвии с настройками
	 				\begin{lstlisting}[language={Java}]
private SimprMessageHandler smartHouse;

@Override
public void start(Stage primaryStage)
		throws Exception {
	smartHouse = new SimprMessageHandler("SmartHouse", "Smart House Simulator", this);
	
	StackPane root = new StackPane();
	Scene scene = new Scene(root, 300, 250);
	primaryStage.setTitle("Simpr Test Application");
	primaryStage.setScene(scene);
	primaryStage.show();
}
	 				\end{lstlisting}
	 			\end{ListingEnv}%
 				\FloatBarrier
 				\item Запустить реализованное приложение.
 				\item Запустить симуляцию в SIMPR.	
 			\end{enumerate}
 		
 		
 		\textbf{В целях тестирования} создан простой класс $ com.github.goto1134.simpr.TestApp $, позволяющий отлаживать приложение SIMPR в диалоговом режиме. 
 		
 		После каждого запроса условия появляется диалог, в котором пользователь должен выбрать вариант ответа.
 		
 		После каждого события происходит вывод в лог.
 		
 		По окончании симуляции приложение сообщает об этом в виде диалога.
			
	\section{Вывод}
		Разработана библиотека для реализации клиента программы SIMPR на языке Java. 
		К библиотеке добавлена документация, а также построен пример использования.