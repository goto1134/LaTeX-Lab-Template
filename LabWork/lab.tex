	\section{Цель работы}
		Научиться описывать сценарии работы компонента распределенной системы.
	\section{Порядок выполенения работы}

	
		\subsection{Главные цели}
			Перед началом работы, модуль $ CrosswalkController $ должен быть проинициализирован.
			После инициализации он меняет состояния сфетофора в цикле.
			\begin{enumerate}
				\item S1. Инициализация
				\item S2. Состояние паузы
				\item S3. Цикл переключения светофора
				\item S4. Реакция на поломку
			\end{enumerate}
			\subsubsection{Проверки}
			\begin{enumerate}
				\item Инициализация, 1 цикл переключения состояний перехода, выключение по команде $ Kernel.Stop $
				\item Инициализация, выключение по команде $ Kernel.Stop $
				\item Инициализация, запуск цикла, прерывание цикла неполадкой контроллера светофора, выгрузка
				\item Инициализация, запуск цикла, прерывание цикла отсутствием ответа контроллера светофора, перезапуск, выключение по команде $ Kernel.Stop $
			\end{enumerate}
		\subsection{Описание сценариев}
		\subsubsection{S1. Инициализация}
			Модуль $ CrosswalkController $ включается сообщением $ Kernel.Start $. 
			При его получении модуль находит светофор и переходит в состояние паузы.
			\paragraph{Проверки}
			\begin{enumerate}
				\item После включения модуль выгружается при получении сообщения о выгрузке $ Kernel.Stop $.
				\item После включения модуль переходит в состояние паузы
			\end{enumerate}
		\subsubsection{S2. Состояне паузы}
			В состоянии паузы управляющая логика должна уметь воспринимать команды выгрузиться (сообщение $ Shutdown $) и возобновить работу (сообщение $ Start $)
			\paragraph{Проверки}
				\begin{enumerate}
					\item Выгрузка по команде  $ Kernel.Stop $
					\item Выгрузка по команде $ Shutdown $
					\item Цикл переключения светофора по команде $ Start $
				\end{enumerate}
		\subsubsection{S3. Цикл переключения светофора}
		Состояние перехода изменяется переключением цвета светофоров для пешеходов и для водителей. Переключения осуществляются по определённому циклу с определёнными задержками:
			\begin{enumerate}
				\item 40 секунд горит зелёный свет
				\item 4 секунд горит жёлтый свет
				\item 30 секунд горит красный
				\item после этого цикл повторяется.
			\end{enumerate}
		\paragraph{Подцели}
			\begin{enumerate}
				\item S3.1. Включить зелёный свет на светофоре
				\item S3.2. Проверить ответ
				\item S3.3. Подождать 40 секунд
				\item S3.4. Включить жёлтый свет на светофоре
				\item S3.5. Проверить ответ (= S3.2)
				\item S3.6. Подождать 4 секунды
				\item S3.7. Включить красный свет на светофоре
				\item S3.8. Проверить ответ (= S3.2)
				\item S3.9. Подождать 30 секунд
				\item S3.10. Повторение S3
			\end{enumerate}
		\paragraph{Проверки}
			\begin{enumerate}
				\item После включения и сообщения $ Start $ через 2 секунды цвет светофора должен быть зелёным
				\item  через 42  секунд после включения и сообщения $ Start $ цвет светофора должен быть жёлтым
				\item  через 46 секунд после включения и сообщения $ Start $ цвет светофора должен быть красным
			\end{enumerate}
		\subsubsection{S3.2. Проверка ответа}
			В случае $ Success $, цикл переключения продолжается. В случае $ Fail $ или $ TimeOut $ выполняется отправить сообщение $ ProblemDetected $ и перейти в состояние паузы.
				\paragraph{Проверки}
				\begin{enumerate}
					\item Получен $ Success $, цикл продолжается.
					\item Получен  $ Fail $, вызов S4
					\item Получен  $ TimeOut $, вызов S4
				\end{enumerate}
			
		\subsubsection{S4. Реакция на поломку}
			Отправляет $ ProblemDetected $ контролирующей логике, переходит в состояние паузы.
			\paragraph{Подцели}
				\begin{enumerate}
					\item S3.1. Отправление $ ProblemDetected $
					\item S3.2. переход в состояние паузы
				\end{enumerate}
			\paragraph{Подцели}
				\begin{enumerate}
					\item Отправление $ ProblemDetected $ и переход в состояние паузы
				\end{enumerate}
	\section{Вывод}
			Описаны сценарии работы компонента распределенной системы.