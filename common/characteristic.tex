
{\actuality} 
Модель анализа включает основные классы, необходимые для реализации выделенных вариантов использования, а также возможные связи между классами. Выделяемые классы разбиваются на три разновидности — интерфейсные, управляющие и классы данных. Эти классы представляют собой набор сущностей, в терминах которых работа системы должна представляться пользователям. Они являются понятиями, с помощью которых достаточно удобно объяснять себе и другим происходящее внутри системы, не слишком вдаваясь в детали.

Интерфейсные классы соответствуют устройствам или способам обмена данными между системой и ее окружением, в том числе пользователями.

Классы данных соответствуют наборам данных, описывающих некоторые однотипные сущности внутри системы. Эти сущности являются абстракциями представлений пользователей о данных, с которыми работает система.

Управляющие классы соответствуют алгоритмам, реализующим какие-то значимые преобразования данных в системе и управляющим обменом данными с ее окружением в рамках вариантов использования.

Модель проектирования является детализацией и специализацией модели анализа. Она также состоит из классов, но более четко определенных, с более точным и детальным распределением обязанностей, чем классы модели анализа. Классы модели проектирования должны быть специализированы для конкретной используемой платформы. 

Каждая такая платформа может включать:
\begin{itemize}
	\item операционные системы всех вовлеченных машин;
	\item используемые языки программирования;
	\item интерфейсы и классы конкретных компонентных сред, таких как J2EE, .NET, COM или CORBA;
	\item интерфейсы выбранных для использования систем управления базами данных, СУБД, например, Oracle или MS SQL Server;
	\item используемые библиотеки разработки пользовательского интерфейса, такие как Windows Forms, Swing или Swt в Java, MFC, VCL или Gtk;
	\item интерфейсы взаимодействующих систем и пр.
\end{itemize}

{\aim} Подготовить модель анализа и проектирования.

Для~достижения поставленной цели необходимо было решить следующие {\tasks}:
\begin{enumerate}
  \item Построить диаграмму классов
  \item Построить диаграмму последовательности
  \item Построить диаграмму кооперации
  \item Построить диаграмму деятельности
  \item Построить диаграмму состояний
\end{enumerate}
    

