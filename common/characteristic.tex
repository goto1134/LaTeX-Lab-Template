
{\actuality} Рассматривается новая модель процессов реального времени.  

Ключевые особенности модели: 

Осуществлении переходов в реальном времени (а не в абстрактном, модельном). 

Всегда реализуется максимально возможный параллелизм срабатывания элементов интерфейса. Этот подход позволяет в системе реального времени избавиться от возможности нескольких событий с одинаковым временем; 

Рассматриваемые системы являются динамическими, могут структурно меняться во времени и иметь при функционировании априори не ограниченную сложность. 

«Вычислительными» переходами – каждый переход представляет некоторые вычисления

{\aim} Проектирование программного продукта, позволяющего описывать и моделировать проходящие в реальном времени параллельные вычислительные процессы в динамических системах. 

Для~достижения поставленной цели необходимо было решить следующие {\tasks}:
\begin{enumerate}
  \item Составить требования к реализации формальной модели.
  \item Изучить доступные инструменты реализации и выбрать наиболее подходящий.
  \item Определить общую архитектуру программы.
\end{enumerate}
    

